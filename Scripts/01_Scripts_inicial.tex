% Options for packages loaded elsewhere
\PassOptionsToPackage{unicode}{hyperref}
\PassOptionsToPackage{hyphens}{url}
%
\documentclass[
]{article}
\usepackage{amsmath,amssymb}
\usepackage{iftex}
\ifPDFTeX
  \usepackage[T1]{fontenc}
  \usepackage[utf8]{inputenc}
  \usepackage{textcomp} % provide euro and other symbols
\else % if luatex or xetex
  \usepackage{unicode-math} % this also loads fontspec
  \defaultfontfeatures{Scale=MatchLowercase}
  \defaultfontfeatures[\rmfamily]{Ligatures=TeX,Scale=1}
\fi
\usepackage{lmodern}
\ifPDFTeX\else
  % xetex/luatex font selection
\fi
% Use upquote if available, for straight quotes in verbatim environments
\IfFileExists{upquote.sty}{\usepackage{upquote}}{}
\IfFileExists{microtype.sty}{% use microtype if available
  \usepackage[]{microtype}
  \UseMicrotypeSet[protrusion]{basicmath} % disable protrusion for tt fonts
}{}
\makeatletter
\@ifundefined{KOMAClassName}{% if non-KOMA class
  \IfFileExists{parskip.sty}{%
    \usepackage{parskip}
  }{% else
    \setlength{\parindent}{0pt}
    \setlength{\parskip}{6pt plus 2pt minus 1pt}}
}{% if KOMA class
  \KOMAoptions{parskip=half}}
\makeatother
\usepackage{xcolor}
\usepackage[margin=1in]{geometry}
\usepackage{color}
\usepackage{fancyvrb}
\newcommand{\VerbBar}{|}
\newcommand{\VERB}{\Verb[commandchars=\\\{\}]}
\DefineVerbatimEnvironment{Highlighting}{Verbatim}{commandchars=\\\{\}}
% Add ',fontsize=\small' for more characters per line
\usepackage{framed}
\definecolor{shadecolor}{RGB}{248,248,248}
\newenvironment{Shaded}{\begin{snugshade}}{\end{snugshade}}
\newcommand{\AlertTok}[1]{\textcolor[rgb]{0.94,0.16,0.16}{#1}}
\newcommand{\AnnotationTok}[1]{\textcolor[rgb]{0.56,0.35,0.01}{\textbf{\textit{#1}}}}
\newcommand{\AttributeTok}[1]{\textcolor[rgb]{0.13,0.29,0.53}{#1}}
\newcommand{\BaseNTok}[1]{\textcolor[rgb]{0.00,0.00,0.81}{#1}}
\newcommand{\BuiltInTok}[1]{#1}
\newcommand{\CharTok}[1]{\textcolor[rgb]{0.31,0.60,0.02}{#1}}
\newcommand{\CommentTok}[1]{\textcolor[rgb]{0.56,0.35,0.01}{\textit{#1}}}
\newcommand{\CommentVarTok}[1]{\textcolor[rgb]{0.56,0.35,0.01}{\textbf{\textit{#1}}}}
\newcommand{\ConstantTok}[1]{\textcolor[rgb]{0.56,0.35,0.01}{#1}}
\newcommand{\ControlFlowTok}[1]{\textcolor[rgb]{0.13,0.29,0.53}{\textbf{#1}}}
\newcommand{\DataTypeTok}[1]{\textcolor[rgb]{0.13,0.29,0.53}{#1}}
\newcommand{\DecValTok}[1]{\textcolor[rgb]{0.00,0.00,0.81}{#1}}
\newcommand{\DocumentationTok}[1]{\textcolor[rgb]{0.56,0.35,0.01}{\textbf{\textit{#1}}}}
\newcommand{\ErrorTok}[1]{\textcolor[rgb]{0.64,0.00,0.00}{\textbf{#1}}}
\newcommand{\ExtensionTok}[1]{#1}
\newcommand{\FloatTok}[1]{\textcolor[rgb]{0.00,0.00,0.81}{#1}}
\newcommand{\FunctionTok}[1]{\textcolor[rgb]{0.13,0.29,0.53}{\textbf{#1}}}
\newcommand{\ImportTok}[1]{#1}
\newcommand{\InformationTok}[1]{\textcolor[rgb]{0.56,0.35,0.01}{\textbf{\textit{#1}}}}
\newcommand{\KeywordTok}[1]{\textcolor[rgb]{0.13,0.29,0.53}{\textbf{#1}}}
\newcommand{\NormalTok}[1]{#1}
\newcommand{\OperatorTok}[1]{\textcolor[rgb]{0.81,0.36,0.00}{\textbf{#1}}}
\newcommand{\OtherTok}[1]{\textcolor[rgb]{0.56,0.35,0.01}{#1}}
\newcommand{\PreprocessorTok}[1]{\textcolor[rgb]{0.56,0.35,0.01}{\textit{#1}}}
\newcommand{\RegionMarkerTok}[1]{#1}
\newcommand{\SpecialCharTok}[1]{\textcolor[rgb]{0.81,0.36,0.00}{\textbf{#1}}}
\newcommand{\SpecialStringTok}[1]{\textcolor[rgb]{0.31,0.60,0.02}{#1}}
\newcommand{\StringTok}[1]{\textcolor[rgb]{0.31,0.60,0.02}{#1}}
\newcommand{\VariableTok}[1]{\textcolor[rgb]{0.00,0.00,0.00}{#1}}
\newcommand{\VerbatimStringTok}[1]{\textcolor[rgb]{0.31,0.60,0.02}{#1}}
\newcommand{\WarningTok}[1]{\textcolor[rgb]{0.56,0.35,0.01}{\textbf{\textit{#1}}}}
\usepackage{graphicx}
\makeatletter
\def\maxwidth{\ifdim\Gin@nat@width>\linewidth\linewidth\else\Gin@nat@width\fi}
\def\maxheight{\ifdim\Gin@nat@height>\textheight\textheight\else\Gin@nat@height\fi}
\makeatother
% Scale images if necessary, so that they will not overflow the page
% margins by default, and it is still possible to overwrite the defaults
% using explicit options in \includegraphics[width, height, ...]{}
\setkeys{Gin}{width=\maxwidth,height=\maxheight,keepaspectratio}
% Set default figure placement to htbp
\makeatletter
\def\fps@figure{htbp}
\makeatother
\setlength{\emergencystretch}{3em} % prevent overfull lines
\providecommand{\tightlist}{%
  \setlength{\itemsep}{0pt}\setlength{\parskip}{0pt}}
\setcounter{secnumdepth}{-\maxdimen} % remove section numbering
\ifLuaTeX
  \usepackage{selnolig}  % disable illegal ligatures
\fi
\IfFileExists{bookmark.sty}{\usepackage{bookmark}}{\usepackage{hyperref}}
\IfFileExists{xurl.sty}{\usepackage{xurl}}{} % add URL line breaks if available
\urlstyle{same}
\hypersetup{
  pdftitle={01\_Scripts\_inicial.R},
  pdfauthor={jhona},
  hidelinks,
  pdfcreator={LaTeX via pandoc}}

\title{01\_Scripts\_inicial.R}
\author{jhona}
\date{2023-09-12}

\begin{document}
\maketitle

\begin{Shaded}
\begin{Highlighting}[]
\CommentTok{\# Jonathan de Jesus de la Rosa}
\CommentTok{\# 2176695}
\CommentTok{\# 21/08/20223}

\CommentTok{\# Importar datos de un archivo de excel a la consola de R}
\CommentTok{\# funcion "read.csv"}

\CommentTok{\# Importar {-}{-}{-}{-}{-}{-}{-}{-}{-}{-}{-}{-}{-}{-}{-}{-}{-}{-}{-}{-}{-}{-}{-}{-}{-}{-}{-}{-}{-}{-}{-}{-}{-}{-}{-}{-}{-}{-}{-}{-}{-}{-}{-}{-}{-}{-}{-}{-}{-}{-}{-}{-}{-}{-}{-}{-}{-}{-}{-}{-}{-}{-}{-}{-}}
\FunctionTok{setwd}\NormalTok{(}\StringTok{"C:/REPOSITORIO/Exp\_Met\_Est\_AD2023/Scripts"}\NormalTok{)}
\NormalTok{ocampo }\OtherTok{\textless{}{-}} \FunctionTok{read.csv}\NormalTok{(}\StringTok{"Ema\_Ocampo.csv"}\NormalTok{ , }\AttributeTok{header =}\NormalTok{ T)}
\FunctionTok{head}\NormalTok{(ocampo)}
\end{Highlighting}
\end{Shaded}

\begin{verbatim}
##   X......DIRS DIRR VELS VELR TEMP HR    PB PREC RAD.SOL TEMPCOMB HUMCOMB
## 1         142  140 11.9 18.0 22.2 22 833.3    0       0     23.7      43
## 2         154  156 12.1 18.0 22.5 21 833.4    0       0     24.7      41
## 3         151  139 10.5 19.1 22.9 21 833.5    0       0     25.2      40
## 4         158  157 10.5 18.4 23.4 20 833.5    0       0     26.0      37
## 5         150  139 12.5 18.7 24.0 19 833.5    0       0     25.6      36
## 6         148  152 12.9 20.2 24.5 18 833.5    0       0     25.8      36
##   HUMSUBS TEMPSUBS
## 1       4     21.9
## 2       4     21.8
## 3       4     21.8
## 4       3     21.8
## 5       4     21.8
## 6       4     21.9
\end{verbatim}

\begin{Shaded}
\begin{Highlighting}[]
\FunctionTok{tail}\NormalTok{(ocampo)}
\end{Highlighting}
\end{Shaded}

\begin{verbatim}
##     X......DIRS DIRR VELS VELR TEMP HR    PB PREC RAD.SOL TEMPCOMB HUMCOMB
## 133          90   86 10.3 19.1 15.2 84 832.5    0       0     11.6      86
## 134          96  107 13.8 20.5 15.3 84 832.5    0       3     11.9      85
## 135         108  127 11.4 23.4 15.5 77 832.5    0      16     12.3      86
## 136         168  136  9.6 23.4 16.4 65 832.6    0      45     13.3      88
## 137         170  142  8.5 18.7 17.6 55 832.6    0      79     14.6      87
## 138         172  139 13.5 33.8 18.9  0 832.6    0     115     16.5      86
##     HUMSUBS TEMPSUBS
## 133       4     21.9
## 134       4     21.7
## 135       4     21.7
## 136       4     21.7
## 137       4     21.5
## 138       4     21.5
\end{verbatim}

\begin{Shaded}
\begin{Highlighting}[]
\CommentTok{\# Descriptivas {-}{-}{-}{-}{-}{-}{-}{-}{-}{-}{-}{-}{-}{-}{-}{-}{-}{-}{-}{-}{-}{-}{-}{-}{-}{-}{-}{-}{-}{-}{-}{-}{-}{-}{-}{-}{-}{-}{-}{-}{-}{-}{-}{-}{-}{-}{-}{-}{-}{-}{-}{-}{-}{-}{-}{-}{-}{-}{-}{-}}

\FunctionTok{mean}\NormalTok{(ocampo}\SpecialCharTok{$}\NormalTok{X......DIRS)}
\end{Highlighting}
\end{Shaded}

\begin{verbatim}
## [1] 107.0145
\end{verbatim}

\begin{Shaded}
\begin{Highlighting}[]
\FunctionTok{mean}\NormalTok{(ocampo}\SpecialCharTok{$}\NormalTok{TEMP)}
\end{Highlighting}
\end{Shaded}

\begin{verbatim}
## [1] 20.95
\end{verbatim}

\begin{Shaded}
\begin{Highlighting}[]
\FunctionTok{median}\NormalTok{(ocampo}\SpecialCharTok{$}\NormalTok{TEMP)}
\end{Highlighting}
\end{Shaded}

\begin{verbatim}
## [1] 20.15
\end{verbatim}

\begin{Shaded}
\begin{Highlighting}[]
\FunctionTok{sd}\NormalTok{(ocampo}\SpecialCharTok{$}\NormalTok{TEMP)}
\end{Highlighting}
\end{Shaded}

\begin{verbatim}
## [1] 5.590258
\end{verbatim}

\begin{Shaded}
\begin{Highlighting}[]
\FunctionTok{var}\NormalTok{(ocampo}\SpecialCharTok{$}\NormalTok{TEMP)}
\end{Highlighting}
\end{Shaded}

\begin{verbatim}
## [1] 31.25099
\end{verbatim}

\begin{Shaded}
\begin{Highlighting}[]
\FunctionTok{range}\NormalTok{(ocampo}\SpecialCharTok{$}\NormalTok{TEMP)}
\end{Highlighting}
\end{Shaded}

\begin{verbatim}
## [1] 12.7 29.6
\end{verbatim}

\begin{Shaded}
\begin{Highlighting}[]
\FunctionTok{fivenum}\NormalTok{(ocampo}\SpecialCharTok{$}\NormalTok{TEMP)}
\end{Highlighting}
\end{Shaded}

\begin{verbatim}
## [1] 12.70 15.60 20.15 26.70 29.60
\end{verbatim}

\begin{Shaded}
\begin{Highlighting}[]
\CommentTok{\# Datos vivero {-}{-}{-}{-}{-}{-}{-}{-}{-}{-}{-}{-}{-}{-}{-}{-}{-}{-}{-}{-}{-}{-}{-}{-}{-}{-}{-}{-}{-}{-}{-}{-}{-}{-}{-}{-}{-}{-}{-}{-}{-}{-}{-}{-}{-}{-}{-}{-}{-}{-}{-}{-}{-}{-}{-}{-}{-}{-}{-}{-}}

\FunctionTok{boxplot}\NormalTok{(ocampo}\SpecialCharTok{$}\NormalTok{TEMP,}
        \AttributeTok{col =} \StringTok{"lightgreen"}\NormalTok{,}
        \AttributeTok{ylim=}\FunctionTok{c}\NormalTok{(}\DecValTok{10}\NormalTok{,  }\DecValTok{30}\NormalTok{),}
        \AttributeTok{ylamb =} \StringTok{"TEMP °C"}\NormalTok{,}
        \AttributeTok{main =} \StringTok{"Ema Ocampo"}\NormalTok{,}
        \AttributeTok{horizontal =}\NormalTok{ T)}
\FunctionTok{mtext}\NormalTok{(}\StringTok{"Ema"}\NormalTok{, }\AttributeTok{side =} \DecValTok{4}\NormalTok{, }\AttributeTok{adj =}\DecValTok{1}\NormalTok{, }\AttributeTok{padj =} \DecValTok{1}\NormalTok{)}
\FunctionTok{mtext}\NormalTok{(}\StringTok{"Datos de agosto 2023"}\NormalTok{, }\AttributeTok{side =}\DecValTok{1}\NormalTok{, }\AttributeTok{padj =}\DecValTok{3}\NormalTok{, }\AttributeTok{adj =}\DecValTok{1}\NormalTok{)}
\end{Highlighting}
\end{Shaded}

\includegraphics{01_Scripts_inicial_files/figure-latex/unnamed-chunk-1-1.pdf}

\begin{Shaded}
\begin{Highlighting}[]
\CommentTok{\# Datos vivero {-}{-}{-}{-}{-}{-}{-}{-}{-}{-}{-}{-}{-}{-}{-}{-}{-}{-}{-}{-}{-}{-}{-}{-}{-}{-}{-}{-}{-}{-}{-}{-}{-}{-}{-}{-}{-}{-}{-}{-}{-}{-}{-}{-}{-}{-}{-}{-}{-}{-}{-}{-}{-}{-}{-}{-}{-}{-}{-}{-}}

\NormalTok{IE }\OtherTok{\textless{}{-}} \FunctionTok{read.csv}\NormalTok{(}\StringTok{"Vivero\_IE.csv"}\NormalTok{, }\AttributeTok{header =}\NormalTok{ T)}
\NormalTok{IE}\SpecialCharTok{$}\NormalTok{Tratamiento }\OtherTok{\textless{}{-}}\FunctionTok{as.factor}\NormalTok{(IE}\SpecialCharTok{$}\NormalTok{Tratamiento)}

\FunctionTok{mean}\NormalTok{(IE}\SpecialCharTok{$}\NormalTok{IE)}
\end{Highlighting}
\end{Shaded}

\begin{verbatim}
## [1] 0.8371429
\end{verbatim}

\begin{Shaded}
\begin{Highlighting}[]
\FunctionTok{tapply}\NormalTok{(IE}\SpecialCharTok{$}\NormalTok{IE, IE}\SpecialCharTok{$}\NormalTok{Tratamiento, mean)}
\end{Highlighting}
\end{Shaded}

\begin{verbatim}
##      Ctrl      Fert 
## 0.7676190 0.9066667
\end{verbatim}

\begin{Shaded}
\begin{Highlighting}[]
\FunctionTok{tapply}\NormalTok{(IE}\SpecialCharTok{$}\NormalTok{IE, IE}\SpecialCharTok{$}\NormalTok{Tratamiento, length)}
\end{Highlighting}
\end{Shaded}

\begin{verbatim}
## Ctrl Fert 
##   21   21
\end{verbatim}

\begin{Shaded}
\begin{Highlighting}[]
\CommentTok{\# Normalidad de datos  {-}{-}{-}{-}{-}{-}{-}{-}{-}{-}{-}{-}{-}{-}{-}{-}{-}{-}{-}{-}{-}{-}{-}{-}{-}{-}{-}{-}{-}{-}{-}{-}{-}{-}{-}{-}{-}{-}{-}{-}{-}{-}{-}{-}{-}{-}{-}{-}{-}{-}{-}{-}}


\CommentTok{\# Shapiro wilks {-}{-}{-}{-}{-}{-}{-}{-}{-}{-}{-}{-}{-}{-}{-}{-}{-}{-}{-}{-}{-}{-}{-}{-}{-}{-}{-}{-}{-}{-}{-}{-}{-}{-}{-}{-}{-}{-}{-}{-}{-}{-}{-}{-}{-}{-}{-}{-}{-}{-}{-}{-}{-}{-}{-}{-}{-}{-}{-}}

\FunctionTok{shapiro.test}\NormalTok{(IE}\SpecialCharTok{$}\NormalTok{IE)}
\end{Highlighting}
\end{Shaded}

\begin{verbatim}
## 
##  Shapiro-Wilk normality test
## 
## data:  IE$IE
## W = 0.96225, p-value = 0.1777
\end{verbatim}

\begin{Shaded}
\begin{Highlighting}[]
\CommentTok{\# Homogeneidad de varianzas {-}{-}{-}{-}{-}{-}{-}{-}{-}{-}{-}{-}{-}{-}{-}{-}{-}{-}{-}{-}{-}{-}{-}{-}{-}{-}{-}{-}{-}{-}{-}{-}{-}{-}{-}{-}{-}{-}{-}{-}{-}{-}{-}{-}{-}{-}{-}}

\FunctionTok{bartlett.test}\NormalTok{(IE}\SpecialCharTok{$}\NormalTok{IE }\SpecialCharTok{\textasciitilde{}}\NormalTok{ IE}\SpecialCharTok{$}\NormalTok{Tratamiento)}
\end{Highlighting}
\end{Shaded}

\begin{verbatim}
## 
##  Bartlett test of homogeneity of variances
## 
## data:  IE$IE by IE$Tratamiento
## Bartlett's K-squared = 3.7423, df = 1, p-value = 0.05305
\end{verbatim}

\begin{Shaded}
\begin{Highlighting}[]
\CommentTok{\# Aplicar la prueba de t para muestras independiente {-}{-}{-}{-}{-}{-}{-}{-}{-}{-}{-}{-}{-}{-}{-}{-}{-}{-}{-}{-}{-}{-}}

\FunctionTok{t.test}\NormalTok{(IE}\SpecialCharTok{$}\NormalTok{IE }\SpecialCharTok{\textasciitilde{}}\NormalTok{ IE}\SpecialCharTok{$}\NormalTok{Tratamiento, }\AttributeTok{var.equal =}\NormalTok{ T)}
\end{Highlighting}
\end{Shaded}

\begin{verbatim}
## 
##  Two Sample t-test
## 
## data:  IE$IE by IE$Tratamiento
## t = -2.9813, df = 40, p-value = 0.004868
## alternative hypothesis: true difference in means between group Ctrl and group Fert is not equal to 0
## 95 percent confidence interval:
##  -0.23331192 -0.04478332
## sample estimates:
## mean in group Ctrl mean in group Fert 
##          0.7676190          0.9066667
\end{verbatim}

\begin{Shaded}
\begin{Highlighting}[]
\CommentTok{\# Prueba sw t de una muestra}
\CommentTok{\#Subconjunto con los datos de ctrl y Fert}

\NormalTok{Ctrl }\OtherTok{\textless{}{-}} \FunctionTok{subset}\NormalTok{(IE}\SpecialCharTok{$}\NormalTok{IE, IE}\SpecialCharTok{$}\NormalTok{Tratamiento }\SpecialCharTok{==} \StringTok{"Ctrl"}\NormalTok{)}
\NormalTok{Fert }\OtherTok{\textless{}{-}} \FunctionTok{subset}\NormalTok{(IE}\SpecialCharTok{$}\NormalTok{IE, IE}\SpecialCharTok{$}\NormalTok{Tratamiento }\SpecialCharTok{==} \StringTok{"Fert"}\NormalTok{)}

\FunctionTok{t.test}\NormalTok{(Ctrl, }\AttributeTok{mu =} \FloatTok{0.95}\NormalTok{)}
\end{Highlighting}
\end{Shaded}

\begin{verbatim}
## 
##  One Sample t-test
## 
## data:  Ctrl
## t = -7.2473, df = 20, p-value = 5.18e-07
## alternative hypothesis: true mean is not equal to 0.95
## 95 percent confidence interval:
##  0.7151253 0.8201128
## sample estimates:
## mean of x 
##  0.767619
\end{verbatim}

\begin{Shaded}
\begin{Highlighting}[]
\FunctionTok{t.test}\NormalTok{(Ctrl, }\AttributeTok{mu =} \FloatTok{0.80}\NormalTok{)                           }
\end{Highlighting}
\end{Shaded}

\begin{verbatim}
## 
##  One Sample t-test
## 
## data:  Ctrl
## t = -1.2867, df = 20, p-value = 0.2129
## alternative hypothesis: true mean is not equal to 0.8
## 95 percent confidence interval:
##  0.7151253 0.8201128
## sample estimates:
## mean of x 
##  0.767619
\end{verbatim}

\begin{Shaded}
\begin{Highlighting}[]
\FunctionTok{t.test}\NormalTok{(Ctrl, }\AttributeTok{mu =} \FloatTok{0.90}\NormalTok{)        }
\end{Highlighting}
\end{Shaded}

\begin{verbatim}
## 
##  One Sample t-test
## 
## data:  Ctrl
## t = -5.2605, df = 20, p-value = 3.788e-05
## alternative hypothesis: true mean is not equal to 0.9
## 95 percent confidence interval:
##  0.7151253 0.8201128
## sample estimates:
## mean of x 
##  0.767619
\end{verbatim}

\begin{Shaded}
\begin{Highlighting}[]
\NormalTok{aire }\OtherTok{\textless{}{-}}\NormalTok{ airquality}


\CommentTok{\# prueba de t dependientes}
\CommentTok{\# Datos de airquality de la ciudad de NY, USA}
\CommentTok{\# Comparar las variables en datos periodos verano (junio)}
\CommentTok{\# Otoño (septimebre)}

\FunctionTok{boxplot}\NormalTok{(aire}\SpecialCharTok{$}\NormalTok{Ozone }\SpecialCharTok{\textasciitilde{}}\NormalTok{ aire}\SpecialCharTok{$}\NormalTok{Month)}
\end{Highlighting}
\end{Shaded}

\includegraphics{01_Scripts_inicial_files/figure-latex/unnamed-chunk-1-2.pdf}

\begin{Shaded}
\begin{Highlighting}[]
\FunctionTok{boxplot}\NormalTok{(aire}\SpecialCharTok{$}\NormalTok{Temp }\SpecialCharTok{\textasciitilde{}}\NormalTok{ aire}\SpecialCharTok{$}\NormalTok{Month)}
\end{Highlighting}
\end{Shaded}

\includegraphics{01_Scripts_inicial_files/figure-latex/unnamed-chunk-1-3.pdf}

\begin{Shaded}
\begin{Highlighting}[]
\NormalTok{aire}\SpecialCharTok{$}\NormalTok{centi }\OtherTok{\textless{}{-}}\NormalTok{ (aire}\SpecialCharTok{$}\NormalTok{Temp }\SpecialCharTok{{-}} \DecValTok{32}\NormalTok{) }\SpecialCharTok{/} \FloatTok{1.8}
\NormalTok{aire}\SpecialCharTok{$}\NormalTok{centi }\OtherTok{\textless{}{-}} \FunctionTok{round}\NormalTok{((aire}\SpecialCharTok{$}\NormalTok{Temp }\SpecialCharTok{{-}} \DecValTok{32}\NormalTok{) }\SpecialCharTok{/} \FloatTok{1.8}\NormalTok{,}\DecValTok{1}\NormalTok{)}
\FunctionTok{boxplot}\NormalTok{(aire}\SpecialCharTok{$}\NormalTok{centi }\SpecialCharTok{\textasciitilde{}}\NormalTok{ aire}\SpecialCharTok{$}\NormalTok{Month)}
\end{Highlighting}
\end{Shaded}

\includegraphics{01_Scripts_inicial_files/figure-latex/unnamed-chunk-1-4.pdf}

\begin{Shaded}
\begin{Highlighting}[]
\CommentTok{\# Crear un subconjunto solo con los meses de Junio y Sept}
\NormalTok{aire.jun }\OtherTok{\textless{}{-}} \FunctionTok{subset}\NormalTok{(aire, Month }\SpecialCharTok{==} \StringTok{"6"}\NormalTok{)}
\NormalTok{aire.sep }\OtherTok{\textless{}{-}} \FunctionTok{subset}\NormalTok{(aire, Month }\SpecialCharTok{==} \StringTok{"9"}\NormalTok{)}

\FunctionTok{t.test}\NormalTok{(aire.jun}\SpecialCharTok{$}\NormalTok{Wind , aire.sep}\SpecialCharTok{$}\NormalTok{Wind, }\AttributeTok{paired =}\NormalTok{T)}
\end{Highlighting}
\end{Shaded}

\begin{verbatim}
## 
##  Paired t-test
## 
## data:  aire.jun$Wind and aire.sep$Wind
## t = 0.094506, df = 29, p-value = 0.9254
## alternative hypothesis: true mean difference is not equal to 0
## 95 percent confidence interval:
##  -1.788913  1.962246
## sample estimates:
## mean difference 
##      0.08666667
\end{verbatim}

\begin{Shaded}
\begin{Highlighting}[]
\FunctionTok{boxplot}\NormalTok{(aire}\SpecialCharTok{$}\NormalTok{Wind }\SpecialCharTok{\textasciitilde{}}\NormalTok{ aire}\SpecialCharTok{$}\NormalTok{Month)}
\end{Highlighting}
\end{Shaded}

\includegraphics{01_Scripts_inicial_files/figure-latex/unnamed-chunk-1-5.pdf}

\end{document}
